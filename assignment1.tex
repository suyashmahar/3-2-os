\documentclass[a4,11pt]{article}


\usepackage{hyperref}
\usepackage{graphicx}
\usepackage{tabularx}
\usepackage{geometry}

\geometry {
  a4paper,
  top=10mm,
  left=18mm,
  right=18mm,
  bottom=25mm,
}

\hypersetup{
  colorlinks=true,
  linkcolor=blue(pigment),
  filecolor=blue(pigment),      
  urlcolor=blue(pigment),
}

\usepackage{fontspec}

%-----------------------------------------------------------
% Comment the following lines if Adobe Garamond Pro font is
% not available as `AGaramondPro'
\setmainfont{AGaramondPro}[
  UprightFont   = *-Regular,
  ItalicFont    = *-Italic,
  BoldFont      = *-Semibold,
  BoldItalicFont= *-BoldItalic
]
%-----------------------------------------------------------

\title{Types of Operating Systems}
\author{Suyash Mahar \\
            ECE - 16116069
}

\date{\today}

\usepackage{multicol}
\begin{document}
\maketitle

\begin{multicols}{2}
  
  \section{Introduction}
  Operating systems can be classified into multiple types based on the
  different functionalities, common types of operating systems are
  listed below:
  \begin{enumerate}
  \item Interactive Operating System
  \item Real-time Operating System
  \item Network Operating System
  \item Application specific Operating System
  \item Embedded Operating System
  \item Server Operating System
  \item Desktop Operating System
  \item Multi-user Operating System
  \end{enumerate}
  \section{Interactive Operating Systems}
  Interactive Operating Systems are a class of operating systems which
  allows user to interact with the operating system. Interactive
  operating system are either single task or multi-task operating
  system. These operating systems have a either a \textit{command line
    interface} or \textit{graphical style interface} for the user.
  \subsection{Command line interface (CLI) Operating Systems}
  Command line interface is a Operating System interface where user
  enters commands as lines of text which are interpreted and executed
  sequentially.

  \textbf{Example:} MS-DOS

  \subsection{Graphical interface Operating Systems}
  Another class of interactive operating system is the graphical user
  interface (GUI) based operating systems. These operating systems
  provide graphics (icons, menus, windows, etc.) for easier use of the
  operating system.

  \textbf{Example:} Windows, Linux with X server or Wayland compositor.
  
  \section{Real-time Operating Systems}
  Real time operating systems are a class of operating systems that
  are for real-time systems. Real-time systems process data as it
  arrives, any delay introduced by buffers or similar structures is
  avoided. Example of real time systems are digital signal processors
  in audio synthesizers, operating systems monitoring sensor data for
  critical applications like aircraft, neurostimulators, etc.

  \textbf{Example:} FreeRTOS and Xenomai.
  
  \section{Network Operating Systems}
  Network operating system is a kind of operating system which is
  designed for networking devices like routers, switches, modems,
  etc. Network operating systems can be considered to be embedded
  operating systems since they are embedded in network equipments.

  \textbf{Example:} IPOS, PicOS...
  
  \section{Application specific Operating Systems (ASOS)}
  An application specific operating systems is a variation of general
  purpose operating system which provides limited number of
  functionalities for a particular application. Being designed for a
  specific application, the OS has better computational and energy
  performance compared to a GPOS for that specific task.

  \textbf{Example:} IPOS, PGO...
  
  \section{Embedded Operating Systems}
  An embedded operating system is an operating system configured for
  a particular hardware that is embedded within a larger machine. These
  operating systems have stronger I/O capabilities to monitor the
  machine. Embedded operating systems are required to be
  performance-efficient and resource optimum.
  
  \section{Server Operating Systems}

  Server operating systems are a class of operating systems that are
  designed to work efficiently as a computer server. These operating
  systems are more effective in handling network and system resources
  to deliver services and application over a network.

  \textbf{Example:} Windows Server 2012, Red Hat Linux...
  \section{Desktop Operating Systems}
  Desktop operating systems are class of operating systems that are
  designed to be run on desktop computers, these operating system are
  user friendly and includes utilities to increase productivity in
  home and office environment.

  \textbf{Example:} Windows 10, Ubuntu 18.04...
  \section{Multi-user Operating Systems}
  Any operating systems that allows multiple users to share a common
  computing resource are referred to as Multi-user operating
  system. Most modern operating systems (excluding ASOS and embedded
  operating systems) are multi-user operating systems. Multi user
  operating systems brings down the overall establishment cost as it
  allows sharing of computer hardware.

  \textbf{Example:} Ubuntu, Kubuntu, Windows 10...
\end{multicols}
\end{document}
