\documentclass[a4,11pt]{article}


\usepackage{hyperref}
\usepackage{graphicx}
\usepackage{tabularx}
\usepackage{geometry}

\geometry {
  a4paper,
  top=10mm,
  left=18mm,
  right=18mm,
  bottom=25mm,
}

\hypersetup{
  colorlinks=true,
  linkcolor=blue(pigment),
  filecolor=blue(pigment),      
  urlcolor=blue(pigment),
}

\usepackage{fontspec}

% -----------------------------------------------------------
% Comment the following lines if Adobe Garamond Pro font is
% not available as `AGaramondPro'
\setmainfont{AGaramondPro}[
UprightFont   = *-Regular,
ItalicFont    = *-Italic,
BoldFont      = *-Semibold,
BoldItalicFont= *-BoldItalic
]
% -----------------------------------------------------------

\title{-Assignment 4.3-\\Various Possibilities on Creating and
  Terminating a Process}
\author{Suyash Mahar \\
  ECE - 16116069 }

\date{\today}

\begin{document}
\maketitle

\section{Creating a Process}
On creating a process following possiblilities exists:
\begin{enumerate}
\item The parent can continue to execute concurently with its children.
\item The parent waits until some or all of its children have
  terminated.
\end{enumerate}

\subsection*{}

Regarding address space of the child process again two possibilities
exists for the process:

\begin{enumerate}
\item The child process has a duplicate address space from the parent. (\textbf{Unix/Linux} style.)
\item The child process has a new program loaded into it (\textbf{Windows} style).
\end{enumerate}

\section{Terminating a Process}
When a process terminates it returns an exit code which represents the
status of the process when it exited. When this occurs all the
resources allocated to the process are deallocated by the operating
system.

Various possibilities during the termination of a process are listed
below:
\begin{enumerate}
\item \textbf{Cascading Termination}: Some systems terminate a
  process's children as soon as the parent process terminates. This is
  known as cascading termination. This is done regardless of normal or
  abnormal exit of the parent.

\item \textbf{Zombie Process}: (\textit{Unix/Linux}) When a child
  process exits and it's parent process hasn't retrived it's exit
  status using \textbf{\textit{wait()}} syscall the child processes
  becomes a zombie process. Such processes are already terminated but
  their exit status is still maintained by the process table. Such
  processes are however usually short lived, as soon as the parent
  calls \textbf{\textit{wait()}} the exit status is returned and the
  entry from the process table is freed.

\item \textbf{Orphan Process}: When process terminates without waiting
  (using \textbf{\textit{wait()}} syscall) for it's children to
  terminate, the children become orphan process. Linux/Unix in such
  case assigns the \textit{init} process as the new parent of the
  child processes.
\end{enumerate}
\end{document}
