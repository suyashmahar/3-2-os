\documentclass[a4,11pt]{article}


\usepackage{hyperref}
\usepackage{graphicx}
\usepackage{tabularx}
\usepackage{geometry}

\geometry {
  a4paper,
  top=10mm,
  left=18mm,
  right=18mm,
  bottom=25mm,
}

\hypersetup{
  colorlinks=true,
  linkcolor=blue(pigment),
  filecolor=blue(pigment),      
  urlcolor=blue(pigment),
}

\usepackage{fontspec}

% -----------------------------------------------------------
% Comment the following lines if Adobe Garamond Pro font is
% not available as `AGaramondPro'
\setmainfont{AGaramondPro}[
UprightFont   = *-Regular,
ItalicFont    = *-Italic,
BoldFont      = *-Semibold,
BoldItalicFont= *-BoldItalic
]
% -----------------------------------------------------------

\title{-Assignment 4-\\Commands for Creating and Terminating a Process}
\author{Suyash Mahar \\
  ECE - 16116069
}

\date{\today}

\begin{document}
\maketitle


\section{Linux/Unix}

\subsection{Process Creation}
The steps for creating a new process in Unix/Linux are following:
\begin{enumerate}
\item \textbf{\textit{fork()}:} First step in creating a new process is to
  call the \textbf{\textit{fork()}} syscall. This system call creates a new
  copy of the parent process. This new child process is the exact
  copy of the parent process.
\item \textbf{\textit{exec()}:} Next step in the process is to replace the
  program in the child by the target program.
\end{enumerate}

\subsection{Process Termination}
Termination of a process is a single system call in Linux/Unix. Both of which provides the \textbf{\textit{exit()}} syscall. This system call takes an exit code as a parameter which is returned to the parent process.

\section{Windows}

\subsection{Process Creation}
The steps for creating a new process in Windows is to use the system call \textbf{\textit{CreateProcess()}}, this system call expects a few parameters including the path of the program to execute. This is unlike the \textbf{\textit{fork()}} system call from Unix/Linux which copies the address space and clones the parent.

\subsection{Process Termination}
There are multiple ways to terminate a process in Windows:

\begin{enumerate}
\item The first way is to call the \textbf{\textit{ExitProcess}}
  system call which terminates the current process.
\item Another way to terminate the proccess is to kill it from a different process using \textbf{\textit{TerminateProcess}} system call. 
\end{enumerate}
\end{document}
